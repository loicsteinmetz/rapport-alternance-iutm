\documentclass[a4paper, 12pt]{report}
\usepackage[utf8]{inputenc}    
\usepackage[T1]{fontenc}
\usepackage[french]{babel}
\usepackage{hyperref}
\usepackage{setspace}
\usepackage{newtxtext,newtxmath}
\usepackage[top=3cm, bottom=3cm, left=3cm, right=3cm]{geometry}
\usepackage[explicit]{titlesec}
\usepackage{graphicx}
\usepackage[stable]{footmisc}
\usepackage{wrapfig}
\usepackage{multicol}
\usepackage{minitoc}
\usepackage{plantuml}

%* Config

\titleformat{\chapter}[display]{}{}{0pt} {
  \parbox{\textwidth} {
    \textbf{\fontsize{30}{30}\selectfont\thechapter.\hspace{0.5cm}\LARGE#1}\\
    \rule{\textwidth}{0.4pt}
  }
}
\titleformat{name=\chapter,numberless}[display]{}{}{0pt} {
  \parbox{\textwidth}{
    \LARGE\textbf{#1}\\
    \rule{\textwidth}{0.4pt}
  }
}

\hypersetup{
    colorlinks=true,
    urlcolor=black,
    linkcolor= black,
    citecolor=black
}

\mtcsetrules{*}{off}

\begin{document}
\renewcommand \partname{\thispagestyle{empty}}
\doparttoc

%* Page de garde

\thispagestyle{empty}
\begin{center}  
\begin{multicols}{2}
  \flushleft
  \null
  \includegraphics[width=\columnwidth]{../res/logo-iut.png}
  \flushright
  \null
  \vspace{0.3cm}
  \large{Promotion 2020-2021}
\end{multicols}
\vspace{2cm}
\LARGE{\textit{Mise en place de traitements massifs et asynchrones dans le cadre de l'import de relèves}}\\
\vspace{2cm}
\large{\textbf{Mémoire présenté en septembre 2021 par}}\\
\vspace{0.5cm}
\Large{\textit{Loïc STEINMETZ}}\\
\vspace{2cm}
\large{\textbf{En vue de l'obtention de la Licence Professionnelle}}\\
\vspace{0.5cm}
\fbox{
  \begin{minipage}{12cm}
    \begin{center}
      \null
      \vspace{0.3cm}
      \textbf{Métiers de l'Informatique}\\
      Conception, développement et test de logiciels\\
      Parcours Métiers du Génie Logiciel\\
      \null
    \end{center}
  \end{minipage}
}
\vspace{2.8cm}
\begin{multicols}{2}
  \flushleft
  \null
  \textbf{Alternance effectuée à :}\\
  Efluid\\
  2 bis, rue Ardant du Picq\\
  CS 10100 57004 METZ Cedex 01\\
  \flushright
  \null
  \includegraphics[width=.6\columnwidth]{../res/logo-efluid.jpg}
\end{multicols}
\end{center}

\clearpage
\thispagestyle{empty}
\null
\clearpage

\thispagestyle{empty}
\begin{center}  
\begin{multicols}{2}
  \flushleft
  \null
  \includegraphics[width=\columnwidth]{../res/logo-iut.png}
  \flushright
  \null
  \vspace{0.3cm}
  \large{Promotion 2020-2021}
\end{multicols}
\vspace{2cm}
\LARGE{\textit{Mise en place de traitements massifs et asynchrones dans le cadre de l'import de relèves}}\\
\vspace{2cm}
\large{\textbf{Mémoire présenté en septembre 2021 par}}\\
\vspace{0.5cm}
\Large{\textit{Loïc STEINMETZ}}\\
\vspace{2cm}
\large{\textbf{En vue de l'obtention de la Licence Professionnelle}}\\
\vspace{0.5cm}
\fbox{
  \begin{minipage}{12cm}
    \begin{center}
      \null
      \vspace{0.3cm}
      \textbf{Métiers de l'Informatique}\\
      Conception, développement et test de logiciels\\
      Parcours Métiers du Génie Logiciel\\
      \null
    \end{center}
  \end{minipage}
}
\vspace{2.8cm}
\begin{multicols}{2}
  \flushleft
  \null
  \textbf{Alternance effectuée à :}\\
  Efluid\\
  2 bis, rue Ardant du Picq\\
  CS 10100 57004 METZ Cedex 01\\
  \flushright
  \null
  \includegraphics[width=.6\columnwidth]{../res/logo-efluid.jpg}
\end{multicols}
\end{center}

%* Remerciements

\chapter*{Remerciements}
\addcontentsline{toc}{chapter}{Remerciements}
\thispagestyle{empty}

Je remercie tout particulièrement mon maître de stage, Alexandre L'HUILLIER, pour m'avoir fait bénéficier de ses compétences tout au long de mon année d'alternance. Merci à lui pour son professionalisme, sa confiance et sa disponibilité.\\

Je remercie également Jean-Luc THIRY, chef de service et Didier GRZEJSZCZAK, chef de filière, pour leur accompagnement général et pour ma bonne intégration au sein de l'entreprise.\\

Mes remerciements vont enfin à l'ensemble des membres du service auquel j'ai été intégré et que je n'ai pas encore cités : Patrice FRANTZ, Julien KEMPF, Fabienne MAMET, François MOLIN, Roderick PIERRE, Stéphane POIROT et Stéphane ROEDEL.

%* Table des matières : mémoire

\part{Mémoire}
\renewcommand{\clearpage}{}
\chapter*{Sommaire}
\renewcommand\ptctitle{}
\parttoc
\thispagestyle{empty}
\renewcommand{\clearpage}{\newpage}
\clearpage

%* Abstract

\chapter*{Abstract}
\addcontentsline{toc}{chapter}{Abstract}

%* Introduction

\chapter*{Introduction}
\addcontentsline{toc}{chapter}{Introduction}

%* Présentation de l'entreprise

\chapter{Présentation de l'entreprise}

%* Sujets traités

\chapter{Sujets traités}

\section{Généralités}
\section{Développements courants}
\section{Projet principal}

%* Traitement des sujets

\chapter{Traitement du projet industriel}

\section{Analyses préliminaires}
\section{Formalisation du besoin et planification}
\section{Conception techniques}
\section{Développements fonctionnels}
\section{Tests}

%* Conclusion

\chapter*{Conclusion}
\addcontentsline{toc}{chapter}{Conclusion}

%* Table des matières : annexes

\part{Annexes}
\renewcommand{\clearpage}{}
\chapter*{Table des annexes}
\renewcommand\ptctitle{}
\parttoc
\thispagestyle{empty}
\renewcommand{\clearpage}{\newpage}
\appendix

%* Annexe : Répartition des parts Efluid

\chapter{Efluid : répartition des parts}
\label{appendix:efluid-parts}

\begin{center}
  \begin{plantuml}
    @startuml

    skinparam monochrome true

    rectangle UEM
    rectangle CDC
    rectangle Enedis
    rectangle Efluid
    
    UEM -- Efluid : 60%
    CDC -- Efluid : 10%
    Enedis -- Efluid : 30%

    @enduml
  \end{plantuml}
\end{center}

\vspace{1cm}
\noindent\textbf{UEM :} Usine d'Electricité de Metz\\
\noindent\textbf{CDC :} Caisse des Dépôts et Consignations

%* Annexe : Organisation des services Efluid

\chapter{Efluid : organisation des services}
\label{appendix:efluid-organisation}

\begin{center}
  \begin{plantuml}
    @startuml

    skinparam monochrome true
    left to right direction

    rectangle Présidence as P
    rectangle "Direction informatique" as DI
    rectangle "Direction commerciale" as DC
    rectangle Ginko as G
    rectangle Technologies as T
    rectangle "Développement : CRM et Facturation" as DFac
    rectangle "Développement : Comptage et réseaux" as DCpt #BBB
    rectangle "Expertise fonctionnelle : CRM et Facturation" as EFac
    rectangle "Expertise fonctionnelle : Comptage et réseaux" as ECpt
    rectangle "Coordination et clients" as CC
    rectangle "Prestation et projets" as PP

    P -- DC
    P -- DI
    P -- G
    DI -- T
    DI -- DFac
    DI -- DCpt
    DC -- EFac
    DC -- ECpt
    DC -- CC
    DC -- PP

    @enduml
  \end{plantuml}
\end{center}

%* Annexe : Volumétrie

\chapter{Volumétrie}
\label{appendix:volumetrie}

\textit{Volumes estimés le 20/01/2021 à partir des copies des bases de données de production.}\\

Les données suivantes permettent d’évaluer le nombre de relèves concernées par le processus de demande de publication quel que soit leur contexte de création.\\

\begin{itemize}
  \item \textbf{UEM}
  \begin{itemize}
    \item \approx\space 3 500 relèves impliquant une création d'échange
    \item \approx\space 91 000 relèves créées
    \item Ratio demande de publication / création de relève : \approx\space 4\%
  \end{itemize}
  \item \textbf{Enedis IDF}
  \begin{itemize}
    \item \approx\space 6 300 000 relèves impliquant une création d'échange
    \item \approx\space 6 400 000 relèves créées
    \item Ratio demande de publication / création de relève : \approx\space 99\%
  \end{itemize}
  \item \textbf{Enedis Est}
  \begin{itemize}
    \item \approx\space 3 000 000 relèves impliquant une création d'échange
    \item \approx\space 3 100 000 relèves créées
    \item Ratio demande de publication / création de relève : \approx\space 95\%
  \end{itemize}
  \item \textbf{Enedis Ouest}
  \begin{itemize}
    \item \approx\space 5 200 000 relèves impliquant une création d'échange
    \item \approx\space 5 400 000 relèves créées
    \item Ratio demande de publication / création de relève : \approx\space 96\%
  \end{itemize}
  \item \textbf{Enedis Méditerranée}
  \begin{itemize}
    \item \approx\space 5 200 000 relèves impliquant une création d'échange
    \item \approx\space 5 400 000 relèves créées
    \item Ratio demande de publication / création de relève : \approx\space 96\%
  \end{itemize}
\end{itemize}
\vspace{0.6cm}

Les données suivantes permettent d’évaluer le nombre de relèves concernées par le processus de demande de publication en ne considérant que les relèves réalisées via un compteur \textit{\textit{Linky}}.\\

\begin{itemize}
  \item \textbf{Enedis IDF}
  \begin{itemize}
    \item \approx\space 5 400 000 relèves de compteurs \textit{\textit{Linky}} impliquant une création d'échange
    \item \approx\space 5 500 000 relèves de compteur \textit{\textit{Linky}} créées
    \item Ratio demande de publication / création de relève : \approx\space 98\%
    \item \approx\space 85\% des relèves créées concernent un compteur \textit{\textit{Linky}}
    \item \approx\space 85\% des demandes de publication de la relève concernent une relève de compteur \textit{\textit{Linky}}
  \end{itemize}
  \item \textbf{Enedis Est}
  \begin{itemize}
    \item \approx\space 2 500 000 relèves de compteurs \textit{Linky} impliquant une création d'échange
    \item \approx\space 2 600 000 relèves de compteur \textit{Linky} créées
    \item Ratio demande de publication / création de relève : \approx\space 95\%
    \item \approx\space 89\% des relèves créées concernent un compteur \textit{Linky}
    \item \approx\space 85\% des demandes de publication de la relève concernent une relève de compteur \textit{Linky}
  \end{itemize}
  \item \textbf{Enedis Ouest}
  \begin{itemize}
    \item \approx\space 4 600 000 relèves de compteurs \textit{Linky} impliquant une création d'échange
    \item \approx\space 4 800 000 relèves de compteur \textit{Linky} créées
    \item Ratio demande de publication / création de relève : \approx\space 96\%
    \item \approx\space 89\% des relèves créées concernent un compteur \textit{Linky}
    \item \approx\space 90\% des demandes de publication de la relève concernent une relève de compteur \textit{Linky}
  \end{itemize}
  \item \textbf{Enedis Méditerranée}
  \begin{itemize}
    \item \approx\space 4 800 000 relèves de compteurs \textit{Linky} impliquant une création d'échange
    \item \approx\space 4 900 000 relèves de compteur \textit{Linky} créées
    \item Ratio demande de publication / création de relève : \approx\space 97\%
    \item \approx\space 90\% des relèves créées concernent un compteur \textit{Linky}
    \item \approx\space 91\% des demandes de publication de la relève concernent une relève de compteur \textit{Linky}
  \end{itemize}
\end{itemize}

%* Annexe : Planification

\chapter{Planification}
\label{appendix:planification}

\begin{itemize}
  \item \textbf{Planification annuelle :}\\
\end{itemize}

\begin{center}
  \begin{plantuml}
    @startuml

    skinparam monochrome true
    scale 1024 width
    scale 768 height

    printscale monthly zoom 1.4
    project starts the 2020-09-01

    [Pré-conception] starts 2020-09-01
    [Pré-conception] ends 2020-12-31
    [Conception] starts 2020-12-01
    [Conception] ends 2021-05-31
    [Développements] starts 2021-03-01
    [Développements] ends 2021-08-31

    @enduml
  \end{plantuml}
\end{center}
\vspace{1cm}

\begin{itemize}
  \item \textbf{Planification de l'implémentation :}\\
\end{itemize}

\begin{center}
  \begin{plantuml}
    @startuml

    skinparam monochrome true
    scale 1024 width
    scale 768 height

    printscale weekly
    project starts the 2021-02-01

    [Formalisation du besoin] starts 2021-02-01
    [Formalisation du besoin] ends 2021-02-15
    --
    [Intégration de la solution] starts 2021-02-08
    [Intégration de la solution] ends 2021-03-07
    --
    [Squelette batch] starts 2021-03-01
    [Squelette batch] ends 2021-03-15
    --
    [Implémentation des chargements] starts 2021-03-16
    [Implémentation des chargements] ends 2021-05-01
    --
    [Implémentation des process] starts 2021-04-16
    [Implémentation des process] ends 2021-06-01
    --
    [Tests] starts 2021-06-01
    [Tests] ends 2021-07-31

    [Squelette batch]->[Implémentation des chargements]
    [Implémentation des chargements]->[Tests]
    [Implémentation des process]->[Tests]
    [Intégration de la solution]->[Tests]

    @enduml
  \end{plantuml}
\end{center}

%* Annexe : Process

\chapter{Traitement de la demande de publication}
\label{appendix:process}

\begin{center}
  \begin{plantuml}
    @startuml

    skinparam monochrome true
    left to right direction

    GestionDemandePublicationReleveProcess <|-- GestionDemandePublicationAvecDonneesProcess
    GestionDemandePublicationReleveProcess <|-- GestionDemandePublicationEstimationReleveFacturationProcess
    GestionDemandePublicationReleveProcess <|-- GestionDemandePublicationImportReleveAvecDonneesProcess

    @enduml
  \end{plantuml}
\end{center}
\vspace{0.5cm}

\begin{center}
  \begin{plantuml}
    @startuml

    skinparam monochrome true
    left to right direction
    scale 250 width

    circle Traitements
    rectangle "Saisie TP" as TP
    rectangle "Auto-relève lors d'une demande de prestation" as AR
    rectangle "CRI avec saisie relève" as CRI
    rectangle REL005MT
    rectangle FAC006MT
    rectangle FAC020MT
    rectangle FAC002MT
    rectangle REL019MT
    rectangle LKY06 #BBB

    Traitements -- TP
    Traitements -- AR
    Traitements -- CRI
    Traitements -- REL005MT
    Traitements -- FAC006MT
    Traitements -- FAC020MT
    Traitements -- FAC002MT
    Traitements -- REL019MT
    Traitements -- LKY06

    @enduml
  \end{plantuml}
\end{center}

%* Tables des matières : général

\tableofcontents
\thispagestyle{empty}

%* Résumé

\chapter*{Résumé}
\thispagestyle{empty}

Résumé.

\end{document}