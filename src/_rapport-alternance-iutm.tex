\documentclass[a4paper, 12pt]{report}
\usepackage[utf8]{inputenc}    
\usepackage[T1]{fontenc}
\usepackage[french]{babel}
\usepackage{hyperref}
\usepackage{setspace}
\usepackage{newtxtext,newtxmath}
\usepackage[top=3cm, bottom=3cm, left=3cm, right=3cm]{geometry}
\usepackage[explicit]{titlesec}
\usepackage{graphicx}
\usepackage[stable]{footmisc}
\usepackage{wrapfig}
\usepackage{multicol}
\usepackage{minitoc}
\usepackage{plantuml}

%* Config

\titleformat{\chapter}[display]{}{}{0pt} {
  \parbox{\textwidth} {
    \textbf{\fontsize{30}{30}\selectfont\thechapter.\hspace{0.5cm}\LARGE#1}\\
    \rule{\textwidth}{0.4pt}
  }
}
\titleformat{name=\chapter,numberless}[display]{}{}{0pt} {
  \parbox{\textwidth}{
    \LARGE\textbf{#1}\\
    \rule{\textwidth}{0.4pt}
  }
}

\hypersetup{
    colorlinks=true,
    urlcolor=black,
    linkcolor= black,
    citecolor=black
}

\mtcsetrules{*}{off}

\begin{document}
\doparttoc
\renewcommand \partname{\thispagestyle{empty}}

%* Page de garde

\thispagestyle{empty}
\begin{center}  
\begin{multicols}{2}
  \flushleft
  \null
  \includegraphics[width=\columnwidth]{../res/logo-iut.png}
  \flushright
  \null
  \vspace{0.3cm}
  \large{Promotion 2020-2021}
\end{multicols}
\vspace{2cm}
\LARGE{\textit{Mise en place de traitements massifs et asynchrones dans le cadre de l'import de relèves}}\\
\vspace{2cm}
\large{\textbf{Mémoire présenté en septembre 2021 par}}\\
\vspace{0.5cm}
\Large{\textit{Loïc STEINMETZ}}\\
\vspace{2cm}
\large{\textbf{En vue de l'obtention de la Licence Professionnelle}}\\
\vspace{0.5cm}
\fbox{
  \begin{minipage}{12cm}
    \begin{center}
      \null
      \vspace{0.3cm}
      \textbf{Métiers de l'Informatique}\\
      Conception, développement et test de logiciels\\
      Parcours Métiers du Génie Logiciel\\
      \null
    \end{center}
  \end{minipage}
}
\vspace{2.8cm}
\begin{multicols}{2}
  \flushleft
  \null
  \textbf{Alternance effectuée à :}\\
  Efluid\\
  2 bis, rue Ardant du Picq\\
  CS 10100 57004 METZ Cedex 01\\
  \flushright
  \null
  \includegraphics[width=.6\columnwidth]{../res/logo-efluid.jpg}
\end{multicols}
\end{center}

\clearpage
\thispagestyle{empty}
\null
\clearpage

\thispagestyle{empty}
\begin{center}  
\begin{multicols}{2}
  \flushleft
  \null
  \includegraphics[width=\columnwidth]{../res/logo-iut.png}
  \flushright
  \null
  \vspace{0.3cm}
  \large{Promotion 2020-2021}
\end{multicols}
\vspace{2cm}
\LARGE{\textit{Mise en place de traitements massifs et asynchrones dans le cadre de l'import de relèves}}\\
\vspace{2cm}
\large{\textbf{Mémoire présenté en septembre 2021 par}}\\
\vspace{0.5cm}
\Large{\textit{Loïc STEINMETZ}}\\
\vspace{2cm}
\large{\textbf{En vue de l'obtention de la Licence Professionnelle}}\\
\vspace{0.5cm}
\fbox{
  \begin{minipage}{12cm}
    \begin{center}
      \null
      \vspace{0.3cm}
      \textbf{Métiers de l'Informatique}\\
      Conception, développement et test de logiciels\\
      Parcours Métiers du Génie Logiciel\\
      \null
    \end{center}
  \end{minipage}
}
\vspace{2.8cm}
\begin{multicols}{2}
  \flushleft
  \null
  \textbf{Alternance effectuée à :}\\
  Efluid\\
  2 bis, rue Ardant du Picq\\
  CS 10100 57004 METZ Cedex 01\\
  \flushright
  \null
  \includegraphics[width=.6\columnwidth]{../res/logo-efluid.jpg}
\end{multicols}
\end{center}

%* Remerciements

\chapter*{Remerciements}
\addcontentsline{toc}{chapter}{Remerciements}
\thispagestyle{empty}

Je remercie tout particulièrement mon maître de stage, Alexandre L'HUILLIER, pour m'avoir fait bénéficier de ses compétences tout au long de mon année d'alternance. Merci à lui pour son professionalisme, sa confiance et sa disponibilité.\\

Je remercie également Jean-Luc THIRY, chef de service et Didier GRZEJSZCZAK, chef de filière, pour leur accompagnement général et pour ma bonne intégration au sein de l'entreprise.\\

Mes remerciements vont enfin à l'ensemble des membres du service auquel j'ai été intégré et que je n'ai pas encore cités : Patrice FRANTZ, Julien KEMPF, Fabienne MAMET, François MOLIN, Roderick PIERRE, Stéphane POIROT et Stéphane ROEDEL.

%* Table des matières : mémoire

\part{Mémoire}
\renewcommand{\clearpage}{}
\chapter*{Sommaire}
\renewcommand\ptctitle{}
\parttoc
\thispagestyle{empty}
\renewcommand{\clearpage}{\newpage}
\clearpage

%* Abstract

\chapter*{Abstract}
\addcontentsline{toc}{chapter}{Abstract}

\textit{Cf. annexe} \ref{appendix:foo}

%* Introduction

\chapter*{Introduction}
\addcontentsline{toc}{chapter}{Introduction}

%* Présentation de l'entreprise

\chapter{Présentation de l'entreprise}

%* Sujets traités

\chapter{Sujets traités}

\section{Généralités}
\section{Développements courants}
\section{Projet principal}

%* Traitement des sujets

\chapter{Traitement du projet principal}

\section{Analyses préliminaires}
\section{Formalisation du besoin et planification}
\section{Conception techniques}
\section{Développements fonctionnels}
\section{Tests}

%* Conclusion

\chapter*{Conclusion}
\addcontentsline{toc}{chapter}{Conclusion}

%* Table des matières : annexes

\part{Annexes}
\renewcommand{\clearpage}{}
\chapter*{Table des annexes}
\renewcommand\ptctitle{}
\parttoc
\thispagestyle{empty}
\renewcommand{\clearpage}{\newpage}
\appendix

%* Annexes

\chapter{Foo}
\label{appendix:foo}

{
  \sffamily{
    \begin{plantuml}
      @startuml
      skinparam monochrome true
      class Alice {
        id: string
      }
      class Bob
      Alice -- Bob
      @enduml 
    \end{plantuml}
  }
}

%* Tables des matières : général

\tableofcontents
\thispagestyle{empty}

%* Résumé

\chapter*{Résumé}
\thispagestyle{empty}

Résumé.

\end{document}